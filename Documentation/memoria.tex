\documentclass[a4paper,12pt]{report}

\usepackage[T1]{fontenc}
\usepackage{graphicx}
\usepackage[spanish]{babel}
\usepackage[utf8]{inputenc}

\begin{document}
	\begin{titlepage}
		\centering
		{\bfseries\LARGE Universidad Las Palmas de Gran Canaria \par}
		\vspace{1cm}
		{\scshape\Large Facultad de Ingeniería Informática \par}
		\vspace{3cm}
		{\scshape\Huge Asesor Inteligente de ropa en función del clima \par}
		\vfill
		{\Large Autora: \par}
		{\Large Andrea Santana López \par}
		\vfill
		{\Large Enero 2026 \par}
	\end{titlepage} 
	
	\newpage
	\tableofcontents
	
	\newpage
	\section{Introdución}
	Gracias a la gran ayuda de la humanidad nuestra planeta sufré provocando muchos problemas como los 
	siguientes:
	1. La perdida de especies
	2. El calentamiento global
	3. La contaminación de agua y aire
	4. La deforestación y degradación de ecosistemas
	5. la sobreexplotación de los recursos primarios 
	6. El cambio climático

	Este último es el que muchas personas están notando ya que a la hora de decidir el conjunto de prendas 
	y accesorios que van a llevar no solo piensan en el ambiente que van a estar,con quien van a estar,si
	me quedará bien por el físico,etc, sino que piensan el clima que podría estar y claro aunque los
	meteorologos intenten hacer predicciones precisas gracias a las ayudas humanas estos no logran 
	grandes preciones provocando que la gente este muy indecisa antes de salir.
	
	Estas indeciones hacen que muchas personas no quieran salir a la calle o que les de igual que llevar y
	acaben poniendose enfermos o que lleven demasiadas prendas o accesorios para cuidarse del frío,calor,lluvia
	o nieve o cualquier tipo de clima.Es por tanto que se me ha ocurrido desarrollar una aplicación movil donde,
	los usuarios al sacarse una foto puedan ver si las prendas que llevan puestas están bien para salir.
	
	\newpage
	\section{Motivación}
	La motivación de este proyecto surgió cuando se  observó que muchos adolescentes no prestan atención a las indicaciones 
	del clima que dice los meteorólogos en las cadenas de noticias.Este suceso ha provocado  la inspiración de
	crear este asesor inteligente de vestimenta en función del clima para que sepan si sus prendas son 
	adecuadas para todo el día o para el tiempo en que estén en el instituto o para cualquier tipo de 
	ocasión en las que vayan a salir.

	\newpage
	\section{Objetivo de la propuesta}
	Mis objetivos de la propuesta serán que el asesor inteligente en función del clima haga 
	las siguientes funciones:
	1. Detección de la persona usando un modelo preentrenado de YOLO
	2. Segmentación de las prendas usando un modelo de segmentación de YOLO
	3. Clasificación de determinadas prendas o accesorios 
	4. Determinar un diagnóstico en función de la recopilación de datos,es decir,de las prendas que lleva.
	5. Dar el diagnóstico al usuario para que sepa si lleva ropa adecuada o no para salir a la calle.
	
	\newpage
	\section{Descripción técnica del trabajo realizado}
	Para realizar el asesor inteligente primero se ha desarrollado unos mockups para la integración del agente de forma
	intuitiva.
	Segundo se ha desarrollado luego donde estaría el asesor inteligente en el backend de la aplicación.
	Para desarrollar el el asesor inteligente se siguio la siguiente metodología:
	
	Primero se entreno a un modelo preentrando el yolov11.n para la detección de personas para 
	que cuando el usuario subiera una imagen o se sacará una foto con el atuendo que llevase ,el cual por 
	defecto tiene la clase persona que es tiene el id = 0 entonces la detectara.
	
	Por lo que únicamente se creo una función llamada detect_person,donde se le pasa una foto con una persona,
	se  lee con la librería de cv2 calcula la confianza,calcula las coordenadas de la caja delimitadora y halla el id que
	identifica el objeto alcanzado que en nuestro caso es la persona.Para ello es necesario que tenga un porcentaje de confianza
	superior al 50 por ciento y que el id de la clase detectada sea una persona.Tras hallar a la persona esta imagen con el bounding 
	box hecho se le lleva a otra función para recortarla donde esta la persona quitando aspectos del fondo.
	
	Segundo se entreno a un modelo de detección de determinados tipos de prenda usando un modelo preentrenado yolov11 usando
	el dataset llamado fashion-detector elaborado por el autor samuel-gonzalez-duran hallado en roboflow.El cual tiene
	la siguientes clases : 'bag', 'dress', 'hat', 'jacket', 'pants', 'shirt', 'shoe', 'shorts', 'skirt' y  'sunglass'.

	Tercero se elaboro una CNN para clasificar el tipo de zapatos para mejorar la predicción y,por último,conectar la api open-meteo
	a nuestro backend a la hora de clasificar los zapatos y detectar la ropa que lleva puesta para estableciendo una serie de parametros 
	el asesor le diga al usuario si las prendas que llevan son correctas o no en función de las predicciones climaticas y estadisticas que
	ha calculado el asesor a través de los datos adquiridos.

	

	
	\newpage
	\section{Fuentes y tecnologías utilizadas}
	Las fuentes que se han utilizado han sido roboflow para la obtención de los datasets de los modelos preentrenados,la fuente oficial de 
	ultralytics,la documentación oficial de la api open-meteo y las tecnologías utilizadas son python,cv2,yolo,cnn. 

	\newpage
	\section{Conclusiones y propuestas de ampliación}
	Como conclusión del proyecto se ha pensado que podría ayudar a gente a decidir más rápido sus prendas y accesorios para salir sin 
	preocuparse de ver noticias o si eres muy depistado.
	En cuanto a propuestas de ampliación sería mejorar el diagnostico estableciendo características de las telas de las prendas y una posible
	integración de voz para decirle a la gente la forma tanto escrito como oral para ayudar a las personas que no son capaces de verse.

	\newpage
	\section{Indicación de herramientas/tecnologías con las que les hubiera gustado contar}
	Me hubiera gustado tener mejor cpu y gpu para el desarrollo para no pegarme muchas horas.

	\newpage
	\section{Diario de reuniones}
	Todos los días desde que se empezo la entrega de 10 a.m a 10.15 a.m
	
	\newpage
	\section{Creditos materiales no originales del grupo}
	1. Fotos
	2. Datasets
\end{document}
